\documentclass{article}
\usepackage[utf8]{inputenc}
\usepackage{graphicx} %imagenes

\title{Sistemas Operativos


Proyecto \\ Thread City

}





\author{Raychell Valeria Arguedas Bolivar\\
Bryan Diaz Barrientos}





\date{Abril 2022}

\renewcommand{\baselinestretch}{1.5}
\begin{document}

\maketitle
  \newpage
  
\section{Introduccion}

En el presente proyecto se debe recrear una presentación de una ciudad llamada Thread City, en la cual vamos a tener hacer una reimplementación de algunas de las funciones de la biblioteca de pthreads del Sistema Operativo GNU/Linux. Haciendo una implementación por medio de la biblioteca pthreads en el espacio de usuario, lo cual nos va a permitir comprender como se puede programar una subsistema del sistema operativo, sin tener la necesidad de hacer cambios en el kernel.\\
Con la implementación de la biblioteca pthread se hará la prueba de esta mediante un programa que utilice nuestra biblioteca.
Vamos tambien a desarrollar un scheduler donde este deba decir cuando se van a crear los hilos, utilizando "turnos".


\newpage


\section{Ambiente de desarrollo}
Como parte del ambiente de desarrollo se esperan hacer uso de los siguientes servicios: \\

\begin{itemize}
\item Rust como lenguaje de programación principal
\item Cargo como su compilador.
\item Visual studio code como editor principal de texto
\item GIT para control de versiones, específicamente GitHub
\item No se ha elegido un Debugger principal, pero se prevé añadir uno tras las versiones iniciales del programa.
\item Uso de la biblioteca GTK
\item Se desarrollará todo el sistema en GNU/Linux. Específicamente una rama hija de Ubuntu, Linux Mint.
\end{itemize}


\newpage
\section{Estructuras de datos usadas y funciones}

\subsection{MyPthreads}
Se realizará la re-implementación de la biblioteca de pthreads, llamada mypthreads, de las siguientes funciones:
\begin{itemize}
\item mythreadcreate
\item mythreadend
\item mythreadyield
\end{itemize}

En las funciones se obtiene un contexto y con las funciones los va modificando.
En el momento en el que un hilo es generado se crea un contexto, el cual va a apuntar a una funcion que le pasan por parametro y luego toma ese contexto y lo pone a funcionar y trabajar. 
Luego de retorno de respuesta pone el contexto base, es decir, donde se encuentra en ese momento.

La funcion de yiel lo que hace es sortear el contexto, es decir, cambia el contexto entre uno y el otro. y el fddidjij lo que hace es devolver al contexto a su punto inicial.

\subsection{Schedulers}
Los schedulers se debe de establecer al momento de crear los threads.
Se logra crear la funcion dentro de mythreacreate donde puede establecer que el scheduler va a utilizar ese hilo.
Nuestra biblioteca de mypthread soporta:
\begin{itemize}
\item Scheduler RoundRobin\\
El RounRobin funciona haciendo uno de nuestra lista dehilos, entonces como tenemos una lista que recibe los hilos, entoces el primero que entra es el primer hilo que se ejecuta, por lo tanto lo que hacer es recorrer toda esa lista
\item Scheduler Sorteo\\
El sorteo funciona generando randoms en base a la lista, es decir usando el largo de la lista para abarcar cada uno de los elementos.\\ Este random va a seleccionar uno de los threads que estan dentro de la lista y lo pone a correr
\end{itemize}



\newpage
\section{Instrucciones para ejecutar el programa}
\begin{itemize}
\item Utilizar un sistema operativo linux, de la rama debian.\\
\item Contar con un ambiente \\
\item Tener cargo nightly preinstalado.\\
\item Contar con las librerias de C, Hstrace, y librerias Python correspondientes.\\
\item Para ejecutar el progra se ben contar con los archivos binarios, lo cuales se encuentran en la carpeta de  stressCMD. \\
\item Recordar que siempre hay que estar en la carpeta de stressCMD.\\
\end{itemize}

\subsection{Comandos de ejecucion}
Si se trabaja dentro de la carpeta del proyecto solo es necesario ejecutar 
\item cargo run

Sino se puede correr el ejecutable guardado en: \\
/target/debug/THREAD-CITY mediante:
\item\textbf{ ./THREAD-CITY}


\subsection{Git Log}
commit 93fe835f1560a960e17893792a56cfa87f48e3d5 (HEAD -> master, origin/master)
Author: brydiaz <bryandiaz13301@gmail.com>
Date:   Sun May 22 21:12:51 2022 -0600

    Intento de progra

commit 62bbd4ea97de86f23b6bd966f9794c8eeb307148
Author: brydiaz <bryandiaz13301@gmail.com>
Date:   Thu Apr 21 23:49:53 2022 -0600

    Tarea terminada(maso menos)

commit 32100f0833b26d1dc5277be5718433ce9a83ec09

\newpage
\section{Actividades realizadas}

Se realiza un total de 37 horas y quizas un poquito mas de trabajo muy aproximado para el proyecto, las cuales estuvieron divididas de la siguiente manera:\\

\item \textbf{Investigación}: 15 horas
\item \textbf{Implementación}: 20 horas 
\item \textbf{Documentación}: 2 horas



\newpage
\section{Evaluacion}
\textbf{MyPthreads:}\\
• Scheduler RoundRobin: 5 \\
• Scheduler Sorteo: 5 \\
• Scheduler en Tiempo Real: 5 \\
• Cambio de Scheduler: 5 \\
• Funciones de la biblioteca pthreads: 10 \\
Documentación utilizando Markdown o Latex-PDF: 20 \\
\textbf{ThreadCity:}\\
• Ambulancias: 10 \\
• Puentes: 15 \\
• Carros: 5 \\
• Plantas nucleares: 15 \\
• Animación: 5 \\
Extra: 10 \\
\newpage
 \section{Autoevaluacion}

Raychell Arguedas \\
[1] [2] [3] [4] [\textbf{5}] Aprendizaje de RR.\\
[\textbf{1}] [2] [3] [4] [5] Aprendizaje de Tiempo Real.\\
[1] [2] [3] [\textbf{4}] [5] Aprendizaje de Lottery.\\
[1] [2] [3] [\textbf{4}] [5] Aprendizaje de pthrads.\\

Bryan Diaz \\
[1] [2] [3] [4] [\textbf{5}] Aprendizaje de RR.\\
[1] [2] [3] [4] [5] Aprendizaje de Tiempo Real.\\
[1] [2] [\textbf{3}] [4] [5] Aprendizaje de Lottery.\\
[1] [2] [3] [4] [\textbf{5}] Aprendizaje de pthrads.\\


\newpage
\section{Lecciones Aprendidas}
Es un hecho de que a lo largo de la carrera y fuera de ella y de todo nos vamos a topar con el uso de hilos, lo cual es de pronto inquitante ya que necesitamos comprender de manera muy minuciosa como funcionan.\\
Aunque fue un proyecto bastante extenso, nos ayuda a tener una mejor vision de el manejo interno de hilos, ya que no es lo mismo poner a trabajar una biblioteca de hilos y solo ponerlos a funcionar, a tener que hacer desde cero una biblioteca completamente. \\
En el scheduler tambien el trabajar con el Round Robin porque de pronto se ha trabajado con metodos parecidos a la hora de trabajar, como para hacer cosas con turnos.
Como proyecto se puede aprender bastante sobre el funcionamiento de hilos, sin embargo se podria aprender de funcionamiento con un proyecto un poco mas llevadero con el peso de curso. 




\newpage

\section{Bibliografia}
https://www.rust-lang.org/es/learn/get-started\\
https://rustwasm.github.io/wasm-pack/book/tutorials/npm-browser-packages/template-deep-dive/cargo-toml.html\\
https://doc.rust-lang.org/book/ch20-02-multithreaded.html\\
https://doc.rust-lang.org/book/ch20-00-final-project-a-web-server.html\\
https://umod.org/plugins/rust-lottery\\
https://www.geeksforgeeks.org/program-round-robin-scheduling-set-1/\\
https://www.javatpoint.com/round-robin-program-in-c\\
https://www.edureka.co/blog/round-robin-scheduling-in-c/\\
https://learnprogramo.com/round-robin-program-in/\\
https://en.wikipedia.org/wiki/Lottery_scheduling#:~:text=Lottery\\
https://www.geeksforgeeks.org/lottery-process-scheduling-in-operating-system/\\
https://github.com/deepak525/Lottery-Scheduling\\
https://mohitesh07.github.io/Lottery-Scheduling/\\
https://developpaper.com/lottery-scheduling-algorithm-let-the-process-work-hard/\\
https://doc.rust-lang.org/std/thread/\\
http://web.mit.edu/rust-lang_v1.25/arch/amd64_ubuntu1404/share/doc/rust/html/book/second-edition/ch16-01-threads.html\\
https://www.koderhq.com/tutorial/rust/concurrency/\\










\end{document}
